%!TEX root=../protocol.tex	% Optional

\section{Einführung}

Diese Übung gibt einen Einblick in Entwicklungen von mobilen Applikationen.

\subsection{Ziele}
Das Ziel dieser Übung ist eine Anbindung einer mobilen Applikation an ein Webservices. 
\\\\
Die Anbindung soll mit Hilfe eines RESTful Webservice umgesetzt werden.

\subsection{Voraussetzungen}

\begin{itemize}
	\item Grundlagen einer höheren Programmiersprache und XML
	\item Grundlegendes Verständnis über Entwicklungs- und Simulationsumgebungen
	\item Verständnis von RESTful Webservices
	\item Download der entsprechenden Entwicklungsumgebung
\end{itemize}

\subsection{Aufgabenstellung}
Es ist eine mobile Anwendung zu implementieren, die sich an das Webservice aus der Übung GK9.3 "Cloud-Datenmanagement" anbinden soll. Dabei müssen die entwickelten Schnittstellen entsprechend angesprochen werden.
\\\\
Es ist freigestellt, welche mobile Implementierungsumgebung dafür gewählt wird. Empfohlen wird aber eine Implementierung auf Android.

\subsection{Bewertung}

\begin{itemize}
	\item Gruppengrösse: 1 Person
	\item Anforderung \textbf{"überwiegend erfüllt"}
	\begin{itemize}
		\item Dokumentation und Beschreibung der angewendeten Schnittstelle
		\item Anbindung einer mobilen Applikation an die Webservice-Schnittstelle
		\item Registrierung von Benutzern
		\item Login und Anzeige einer Willkommensnachricht
	\end{itemize}
	\item Anforderungen \textbf{"zur Gänze erfüllt"}
	\begin{itemize}
		\item Simulation bzw. Deployment auf mobilem Gerät
		\item Überprüfung der funktionalen Anforderungen mittels Regressionstests
	\end{itemize}
\end{itemize}
